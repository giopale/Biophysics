\documentclass[t]{beamer}
\usepackage[italian]{babel}
\usepackage{amsmath}
\usepackage{amsthm}
\usepackage{amssymb}
\usepackage{amscd}
\usepackage{amsfonts}
\usepackage{booktabs}
\usepackage{enumerate}
\usepackage{array}
\usepackage{comment}
\usepackage{animate}
\usepackage{graphicx}
\usepackage{adjustbox}
\usepackage{tcolorbox}
\usepackage[algo2e]{algorithm2e}
\usepackage{bm}
\usepackage{subfig}
\usepackage{tikz}
\usepackage{xcolor}
\usepackage{commath}
\usepackage{siunitx}
\usepackage{physics}
\usepackage{wrapfig}

\newcommand{\ca}{\text{Ca}^{2+}}

\usepackage{etoolbox} %Chiama la barra del titolo anche senza inserire il titolo slide
\setlength{\parskip}{5pt}%

% Choose how your presentation looks.
%
% For more themes, color themes and font themes, see:
% http://deic.uab.es/~iblanes/beamer_gallery/index_by_theme.html
%
\mode<presentation>
{
  \usetheme{default}      % or try Darmstadt, Madrid, Warsaw, ...
  \usecolortheme{default} % or try albatross, beaver, crane, ...
  \usefonttheme{default}  % or try serif, structurebold, ...
  \setbeamertemplate{navigation symbols}{}
  \setbeamertemplate{caption}[numbered]
} 

% \usepackage[utf8]{inputenc}
% \usepackage[T1]{fontenc}

\title{Single Channel Kinetics of BK Channels}
\subtitle{A review article by Y. Geng and K. L. Magleby}
\author{Giorgio Palermo}
\date{08/07/2020}
\institute{LM Physics - a.a. 2019/20 - Biological Physics }

\begin{document}

\begin{frame}{}
  \titlepage
\end{frame}

% Uncomment these lines for an automatically generated outline.
%\begin{frame}{Outline}
%  \tableofcontents{}
%\end{frame}
% \begin{frame}{Aims}
% The aim of this presentation is to give a general description and interpretation of what experiments found about the gating process of BK channels.

% \medskip
% Potassium transfer through the cell membrane is regulated by different kind of channels; among these one special mention goes to the BK channels, which are large conductance calcium-activated potassium channels, hence the name, which stands for \emph{big potassium}.

% \end{frame}

\begin{frame}{Introduction}
The aim of this presentation is to give a general description and interpretation of what experiments found about the gating process of BK channels.

Potassium is one of the most important ions involved in the cell physiology.
Its concentrations inside and outside the cell membranes, together with the ones of other ions, strongly determines the electric properties of the cell membrane, such as the resting potential. 

Potassium transfer through the cell membrane is regulated by different kind of channels; among these one special mention goes to the BK channels, which are large conductance calcium-activated potassium channels, hence the name, which stands for \emph{big potassium}.

\end{frame}

\begin{frame}{Introduction}
One remarkable property of BK channels is that they are part of a negative feedback system that regulates Calcium and Potassium transfer.

This mechanism involves also voltage activated Calcium channels that are typically located in the vicinity of BK channels.

%Remarkably, the voltage dependent activation of the BK channels creates a negative feedback system to drive the membrane potential more negative, which would then close both the open BK channels and also the voltage dependent $\ca$ channels which are often localized in the vicinity.

Through this feedback mechanism BK channels are involved in many physiological processes.

Their dysfunction can lead to diseases such autism, mental retardation, epilepsy and others.

In particular we consider what types of kinetic gating mechanism can account for the single-channel current records for the activation of BK channels by $\ca$ and voltage.
\end{frame}

\begin{frame}{Structure of a BK channel}
We are now looking at a schematic diagram of the modular structure of a BK channel.

Only two of the four subunits are shown with the front and back subunits removed.

Bk channels made of four transmembrane voltage sensor domains (VSD) surrounding a central pore gate domain (PGD), and a large cytosolic domain (CTD) also referred as gating ring.

Each VSD is formed by five transmembrane segments, S0 to S4 of a single subunit.

VSDs are resposible for the voltage sensibility of the channel.

\end{frame}

\begin{frame}{Structure of a BK channel}
The gating ring is formed by the RCK1 and RCK2 (regulators of conductance) domains of each of the four subunits

The RCK modules together form a ring with a central opening (gating ring).

The gating ring is responsible for the $\ca$ sensibility of the channel.

VSDs and RCKs are allosterically coupled to the PGD through linkers.

Through linkers, VSDs and RCKs apply force on the PGD in order to activate the channel.
\end{frame}

\begin{frame}{DSM models}
    Now that we know the modular structure of the BK channels, we introduce a class of models used to describe gating of BK channels: DSM.

    The main idea is that channels gate by moving among conformational states, which in our case are identified by three parameters: num. of activated VSDs, num. of activated RCKs, open/closed PGD.

    In a 2-state model the dwell-time of a state in DSM is exponentially distributed with characteristic time $\tau = 1/k,$ where $k$ is called rate constant for the transition.

    More generally, for multiple state models, the dwell-time of the i-esim state is still exponentially distributed, but with characteristic time $\tau_i = 1/\sum_j k_{ij}.$

    This situation is referred to as \emph{mixed exponentials.}


\end{frame}

\begin{frame}{DSM models}
    There are three crucial assumptions one ch makes describing the kinetics of BK channels through DSM, which are:
    \begin{itemize}
      \item Short transition times
      \item Rate constants invariant for constant experimental conditions
      \item The gating takes place by means of changing the rate constants
    \end{itemize}
    
    It follows from the second assumption that the rate constants for leaving a state are independent of the previous sequence of states.

    For this reason Markov processes are often referred to as \emph{memory less} processes, even though saying that the rate constants are memory less would be more correct.

\end{frame}


\begin{frame}{Gating mechanism for BK channels}
we want to formulate a kinetic model in accordance to what we know about the modular structure of the considered ion channel.

The BK channel structure, as already discussed, is made of a pore gate domain and four voltage sensitive domains plus four pairs of intertwined RCK domains which are properly linked to the PGD to control its gating.

States of the model are identified by different conformational configurations of the modules composing the channel.

Observations say that voltage and $\ca$ activations are mainly independent processes.

Thus is natural to build an overall kinetic model starting with separate models for each.

\end{frame}

\begin{frame}{A scheme for voltage dependence}

Regarding the voltage dependence, the gating mechanism is described by [SCHEME 4].

In this 10-state model, each state is comprised of four subunits, as dictated by the tetrameric structure of BK channels.

The five states in the upper tier are closed and the five states in the lower tier are open, where the open state is indicated by a circle in the middle of the four subunits on the lower tier.

A square subunit indicates that the voltage sensor in that subunit is relaxed (deactivated), whereas a circular subunit indicates that the voltage sensor is activated.
\end{frame}

\begin{frame}{A scheme for $\ca$ dependence} 
The gating mechanism for the $\ca$ dependent gating of BK channels is presented in [SCHEME 5]

This scheme parallels the voltage activation shown before except that the activation is by binding $\ca,$ indicated by a shaded subunit.

It is important to notice that there is not an obligatory coupling between sensor (either voltage or $\ca$) activation and channel opening and closing.

The channel can open with 0 to 4 activated sensors. 

Opening probability increases with the number of activated voltage sensors.


\end{frame}

\begin{frame}{Total scheme}
The combination of the voltage and the $\ca$ dependence is shown in [SCHEME 6].

[SCHEME 6] includes five repetitions of [SCHEME 4] with 0 to 4 bound $\ca$ per channel.

The scheme shows that the channel could potentially enter 25 closed states on the upper tier and 25 open states on the lower tier during gating.

Each state is defined by a different combination of activated voltage sensors and bound $\ca$ sites.
\end{frame}

\begin{frame}{Effective number of visited states (I)}
The figure [figure 6] presents the single channel recordings obtained from a single BK channel at three different $\ca$ concentrations.

What is readily apparent is $\ca$ induced increase in channel activity.


\end{frame}


\begin{frame}{Effective number of visited states (II)}
Measuring the open and closed interval durations from long records of stable data and plotting them as frequency histograms on log-log coordinates we get the open and closed dwell-time distribution for a fixed $\ca.$

The open distributions are well described by the sum of three significant exponential components and the closed distribution by the sum of five significant exponential components.

This is a rather general result, as open distributions are typically described by 3-4 significant exponential components and closed distribution are typically described by 4-6.

The minimum number of exponential components suggests that gating involves a minimum of 3-4 open states and 4-6 closed states.

\end{frame}

\begin{frame}{Effective number of visited states (III)}
Then why isn't the BK gating over 25 open and 25 closed states?

One reason is that the 50 states are potential states based on the modular structure of the BK channels, but any given $\ca$ and voltage, gating would be effectively limited to only a small subset of 50 states.

So only a subset of the 50 states would generate exponentials with sufficiently large areas to detect.

Furthermore, calculations on large multistate models show that many of the expected time constants for 50-state models can be nearly identical or too close to detect.
\end{frame}

\begin{frame}{Activation of BK channels by voltage}
The activation of a single BK channel by voltage at the single channel level is shown in [FIGURE 9a]

Voltages range from -100 to +100 mV with fixed $95 \ \mu\text{M}\ \ca.$ 

Changing the voltage from -100 to +100 mV decreased the mean observed closed time 4500 times, while increasing the observed mean closed time of 12 times.

The action of depolarization is mainly a destabilization of the closed states (reduced lifetime), with a much smaller stabilization of the open states.
\end{frame}

\begin{frame}{Activation of BK channels by $\ca$}
In [FIGURE 6a] we see the activation of a BK channel at the single-channel level by $\ca,$ measured at a fixed +30 mV potential.

The dependence of the opening probability on the $\ca$ concentration is steep.

Increasing the $\ca$ from $0.1$ to $1000\ \mu\text{M}$ increased the mean open interval duration 10 times and decreased mean closed interval duration about 1000 times.

Also in this case the major action of $\ca$ is a reduction of the mean dwell-time of the closed states, indicating their destabilization.

An increase in the mean open dwell-time indicates stabilization of open states, even though much smaller than the closed states destabilization. 

\end{frame}

\begin{frame}{Synergistic activation of BK channels by voltage and $\ca$ (I)}
The joint activation of a BK channel by voltage and $\ca$ is shown in [FIGURE 11].

Jointly increasing of $\ca$ and depolarization leads to synergistic increases in the opening probability, as we can see following the top-left to bottom-right diagonal of the figure.

Increasing $\ca$ gives approximately parallel shifts in the opening probability vs. voltage curves, suggesting independent activation by $\ca$ and voltage, consistently with the modular model discussed before.
\end{frame}

\begin{frame}{Synergistic activation of BK channels by voltage and $\ca$ (II)}
This figures contain a plot opening probability vs. voltage at different concentrations in two different scales.

It's clear that increasing $\ca$ gives approximately parallel shifts in the opening probability vs. voltage curve.

This fact supports the hypothesis of independent activation by $\ca$ and voltage, consistently with the modular model discussed before.
\end{frame}

\begin{frame}{Does the model describe the data?}
The consistency of DSM models was tested on models with reduced number of states, obtaining good to excellent description of the data.

A reduced model lowers the number of constants in such a way it is easier to determine them through simultaneous fitting of data obtained over a range of $\ca$ and voltage w.r.t. a full 50-state model.

The consistency of the 50-state scheme with synergistic activation by voltage and $\ca$ was deduced by the previous result.

The idea is that if the reduced scheme can account for the single-channel kinetics, then the full 50-state scheme is also able to account for the kinetics of the system because it is a generalization of the reduced model.


\end{frame}

% \begin{frame}{On the action of the gating ring}
% To explore the action of the gating ring and the RCK1-S6 linkers, experiments on mutated BK channels have been made.
% In particular the mutations involved the absence of the gating ring and the length of the RCK1-S6 linkers.

% It has been shown that the linkers passively pull on the PGD to bias the channel towards opening in absence of $\ca,$ while they apply an active opening force in  presence of $\ca.$
% \end{frame}

\begin{frame}{Conclusion}
BK channels are high-conductance potassium channels with a tetrametric modular structure.

Their garing mechanism is described by DSM models.

In particular the $\ca$ dependent gating and the voltage dependent gating can each be approximated by a two-tiered 10-state model with five closed states on the upper tier and five open states in the lower tier.

The joint activation by $\ca$ and voltage is obtained with five repeats of the 10-state voltage activation mechanism, where each repeat has 0 to 4 activated $\ca$ sites.

Observations confirm that, to a first approximation, voltage and $\ca$ can act independently to activate the channel.

It has been observed that the channel activation takes place mainly by increasing the opening rate, or equivalently destabilizing the closed states.


\end{frame}

\begin{frame}{Conclusion}
Future studies need to put light on some open problems:

The possibility of interaction among the various sensors arises considering that both VSDs and $\ca$ sensors are attached to the PGD.

The idea to test is if they interact allosterically through it.

A preliminary study has shown that a highly constrained 50-state model can approximate the single-channel data.

A viable model would describe both the single channel kinetics and the kinetics of the macro current data.

Future studies need to expand the BK models to take into account the two high affinity $\ca$ binding sites rather than the usual assumption of one high affinity site.

This would shift the 50-state model to a 250-state model, with increased computational complexity.
\end{frame}


\end{document}








