\documentclass{beamer}
\usepackage[italian]{babel}
\usepackage{amsmath}
\usepackage{amsthm}
\usepackage{amssymb}
\usepackage{amscd}
\usepackage{amsfonts}
\usepackage{booktabs}
\usepackage{enumerate}
\usepackage{array}
\usepackage{comment}
\usepackage{animate}
\usepackage{graphicx}
\usepackage{adjustbox}
\usepackage{tcolorbox}
\usepackage[algo2e]{algorithm2e}
\usepackage{bm}
\usepackage{subfig}
\usepackage{tikz}
\usepackage{xcolor}
\usepackage{commath}
\usepackage{siunitx}
\usepackage{physics}

\newcommand{\ca}{\text{Ca}^{2+}}

\usepackage{etoolbox} %Chiama la barra del titolo anche senza inserire il titolo slide
%
% Choose how your presentation looks.
%
% For more themes, color themes and font themes, see:
% http://deic.uab.es/~iblanes/beamer_gallery/index_by_theme.html
%
\mode<presentation>
{
  \usetheme{default}      % or try Darmstadt, Madrid, Warsaw, ...
  \usecolortheme{default} % or try albatross, beaver, crane, ...
  \usefonttheme{default}  % or try serif, structurebold, ...
  \setbeamertemplate{navigation symbols}{}
  \setbeamertemplate{caption}[numbered]
} 

% \usepackage[utf8]{inputenc}
% \usepackage[T1]{fontenc}

\title{Single Channel Kinetics of BK Channels}
\subtitle{A review article by Y. Geng and K. L. Magleby}
\author{Giorgio Palermo}
\date{08/07/2020}
\institute{LM Physics - a.a. 2019/20 - Biological Physics }

\begin{document}

\begin{frame}{}
  \titlepage
\end{frame}

% Uncomment these lines for an automatically generated outline.
%\begin{frame}{Outline}
%  \tableofcontents{}
%\end{frame}
\begin{frame}{Introduction}
Potassium is one of the most important ions involved in the cell physiology.
Its concentrations inside and outside the cell membranes strongly determines, together with the ones of other important ions, the electric properties of the cell membrane, such as the resting potential.

The potassium transfer through the cell membrane is regulated by different kind of channels; among these one special mention goes to the BK channel, which has the property of being activated by the joint action of depolarization and $\text{Ca}^{2+}.$

\end{frame}

\begin{frame}{Introduction}
Remarkably, the voltage dependent activation of the BK channels creates a negative feedback system to drive the membrane potential more negative, which would then close both the open BK channels and also the voltage dependent $\ca$ channels which are often localized in the vicinity.

Through this feedback mechanism BK channels are involved in many physiological processes and their dysfunction can lead to diseases such autism, mental retardation, epilepsy and others.
\end{frame}

\begin{frame}{Aims}
The aim of this presentation is to give a general description and interpretation of what experiments found about the gating process of BK channels.

In particular we consider what types of kinetic gating mechanism can account for the single-channel current records for the activation of BK channels by $\ca$ and voltage.

\end{frame}

\begin{frame}

\end{frame}

\begin{frame}{Methods}

\end{frame}

\begin{frame}{Figures}

\end{frame}

\begin{frame}{Conclusion}

\end{frame}


\end{document}








